TODO: 5.3, 5.4

\section*{5.1: A chemical process}
%%% AUTHOR: BENOIT SLUYSMANS %%%

In order to construct the state model, let's define the foowwing state variables:\\
\indent $x_1$ = species A concentration in the reactor\\
\indent $x_2$ = species B concentration in the reactor\\
\indent $x_3$ = species C concentration in the reactor\\
\indent $x_4$ = species B concentration in the recycling tank\\
\indent $V_1$ = reactor volume\\
\indent $V_2$ = recycling tank volume\\

By placing adequately the different flows $F_0, F_1, F_2, F_3, F_4$, it directly follows that
\begin{eqnarray*}
\dot{V_1} = F_1+F_2-F_0\\
\dot{V_2} = F_3+F_4-F_2
\end{eqnarray*}

By writing the system with two irreversible reactions, we obtain
\begin{eqnarray*}
A+B\rightarrow C \\
C\rightarrow A+B
\end{eqnarray*}

From there, we can write the stoechiometric matrix $C$ and, using the mass action law, the rate of these reactions $r_1$ and $r_2$:

\begin{equation*}
C = \begin{bmatrix}
-1 & 1\\
-1 & 1\\
1 & -1
\end{bmatrix}
\end{equation*}
and
\begin{eqnarray*}
r_1 &=& k_1 x_1 x_2\\
r_2 &=& k_2 x_3 
\end{eqnarray*}


If $x_A^{in}$ and $x_B^{in}$ are the concentrations of species A and B that enter the system, the balance equations give:
\begin{eqnarray*}
\dot{V_1x_1} &=& (-k_1 x_1 x_2+k_2 x_3 )V_1 + F_1 x_A^{in} - F_0 x_1 \\
\dot{V_1x_2} &=& (-k_1 x_1 x_2+k_2 x_3 )V_1 + F_2 x_4 - F_0 x_2 \\
\dot{V_1x_3} &=& (k_1 x_1 x_2-k_2 x_3 )V_1  - F_0 x_3 \\
\dot{V_2x_4} &=& F_0 x_2 + F_3 x_B^{in} - F_2 x_4
\end{eqnarray*}

As we have a perfect separator, it follows that $F_4 \rho_B = F_0 x_2$, with $\rho_B$ the density of species B. Finally, defining the input variables
\begin{eqnarray*}
u_0 &=& \frac{F_0}{V_1}\\
u_1 &=& \frac{F_1}{V_1}\\
u_2 &=& \frac{F_2}{V_1}\\
u_3 &=& \frac{F_3}{V_2}
\end{eqnarray*}
we obtain the following state model
\begin{eqnarray*}
\dot{x_1} &=& -k_1 x_1 x_2+k_2 x_3 + u_1 (x_A^{in}-x_1)-u_2 x_1\\
\dot{x_2} &=& -k_1 x_1 x_2+k_2 x_3 + u_2 x_4-(u_1+u_2)x_2\\
\dot{x_3} &=& k_1 x_1 x_2-k_2 x_3 -(u_1+u_2)x_3\\
\dot{x_4} &=& u_0\frac{V_1}{V_2}x_2(1-\frac{x_4}{\rho_B})+u_3(x_B^{in}-x_4)\\
\dot{V_1} &=& (u_1+u_2-u_0)V_1\\
\dot{V_2} &=& u_3V_2+(\frac{u_0x_2}{\rho_B}-u_2)V_1
\end{eqnarray*}






\section*{5.2: Reactors with separate power supplies}
%%% AUTHOR: BENOIT SLUYSMANS %%%

\subsection*{1.}
Let's define the input variables
\begin{eqnarray*}
u_1 &=& \frac{F_{01}}{V}\\
u_2 &=& \frac{F_{02}}{V}
\end{eqnarray*}
Remembering that the stoechiometric matrix was
\begin{eqnarray*}
C = \begin{bmatrix}
-1 & 0 \\
-1 & 0 \\
2 & -2 \\
0 & 1
\end{bmatrix}
\end{eqnarray*}
and that the rate of the two reactions were
\begin{eqnarray*}
r_1 &=& k_1x_1x_2e^{(-Kx_4)}\\
r_2 &=& k_2x_3^2
\end{eqnarray*}
we can write the following state model, with $c_1^{in}$ and $c_2^{in}$ the concentrations of species 1 and 2 that enter the system:
\begin{eqnarray*}
\dot{x_1} &=& -k_1x_1x_2e^{(-Kx_4)} + u_1 c_1^{in} - (u_1+u_2)x_1 \\
\dot{x_2} &=& -k_1x_1x_2e^{(-Kx_4)} + u_2 c_2^{in} - (u_1+u_2)x_2 \\
\dot{x_3} &=& 2k_1x_1x_2e^{(-Kx_4)} - 2k_2x_3^2 - (u_1+u_2)x_3 \\
\dot{x_4} &=& k_2x_3^2 - (u_1+u_2)x_4
\end{eqnarray*}
Indeed, the volume $V$ is constant and therefore $F_{out} = F_{01} + F_{02}$.

\subsection*{2.}
Here, $F_{01}+F_{02} = cst = dV \Leftrightarrow u_1 + u_2 = d$, so we only need one input variable. Let's take $u_1=w$ and $u_2=d-w$. We can rewrite
\begin{eqnarray*}
\dot{x_1} &=& -k_1x_1x_2e^{(-Kx_4)} + w c_1^{in} - d x_1 \\
\dot{x_2} &=& -k_1x_1x_2e^{(-Kx_4)} + (d-w) c_2^{in} - dx_2 \\
\dot{x_3} &=& 2k_1x_1x_2e^{(-Kx_4)} - 2k_2x_3^2 - dx_3 \\
\dot{x_4} &=& k_2x_3^2 - dx_4
\end{eqnarray*}
We can show that this model can be written as (5.8):
\begin{eqnarray*}
\frac{dx}{dt} &=& \begin{bmatrix}
-1 & 0 \\
-1 & 0 \\
2 & -2 \\
0 & 1
\end{bmatrix}
\begin{bmatrix}
k_1x_1x_2e^{(-Kx_4)} \\
k_2x_3^2
\end{bmatrix} - d\begin{bmatrix}
x_1\\
x_2\\
x_3\\
x_4
\end{bmatrix} + \begin{bmatrix}
w c_1^{in} \\
(d-w) c_2^{in}\\
0\\
0
\end{bmatrix}  \\
\Leftrightarrow \frac{dx}{dt} &=& C r(x) - q_{out}(x,w) + q_{in}(x,u)
\end{eqnarray*}




\section*{5.3: Gazeous reactives and products}

\section*{5.4: An aerobic biological wastewater treatment plant}





\section*{5.5: A nonconservative system}
%%% AUTHOR: BENOIT SLUYSMANS %%%
\subsection*{1.}
Starting from
\begin{eqnarray*}
X_1\rightarrow X_2+X_3 \\
X_3\rightarrow 2X_1+X_4
\end{eqnarray*}
we can write the stoechiometric matrix
\begin{eqnarray*}
\begin{bmatrix}
-1 & 2 \\
1 & 0 \\
1 & -1 \\
0 & 1
\end{bmatrix}
\end{eqnarray*}
and the rate of these reactions as described in the exercise
\begin{eqnarray*}
r_1 &=& k_1 x_1^2\\
r_2 &=& k_2 \frac{x_3}{K_3+x_3}\frac{K_2}{K_2+x_2}
\end{eqnarray*}
As this is a closed system, we can directly write the following state model
\begin{eqnarray*}
\dot{x_1} &=& - k_1 x_1^2 + 2k_2 \frac{x_3}{K_3+x_3}\frac{K_2}{K_2+x_2} \\
\dot{x_2} &=& k_1 x_1^2  \\
\dot{x_3} &=& k_1 x_1^2 - k_2 \frac{x_3}{K_3+x_3}\frac{K_2}{K_2+x_2} \\
\dot{x_4} &=& k_2 \frac{x_3}{K_3+x_3}\frac{K_2}{K_2+x_2}
\end{eqnarray*}


\subsection*{2.}
The system is conservative if $\exists w>0 | C^Tw = 0$, which is equivalent to
\begin{eqnarray*}
\begin{bmatrix}
-1 & 1 & 1 &0\\
2 & 0 & -1 & 1
\end{bmatrix}
\begin{bmatrix}
w_1 \\
w_2\\
w_3\\
w_4
\end{bmatrix} = \begin{bmatrix}
-w_1+w_2+w_3 \\
2w_1-w_3+w_4
\end{bmatrix} = \begin{bmatrix}
0 \\
0
\end{bmatrix}
\end{eqnarray*}
If we sum these two equations, we have $w_1+w_2+w_4=0$. That condition is impossible with $w_i > 0$. We can conclude that this system is nonconservative. Intuitively, we can see that one $X_1$ gives one $X_3$, and one $X_3$ gives two $X_1$, so the system would explode.

\subsection*{3.}
To make this system conservative, we can add a reactive to the first reaction, in order to modify $C$:
\begin{eqnarray*}
X_0 + X_1\rightarrow X_2+X_3 \\
X_3\rightarrow 2X_1+X_4
\end{eqnarray*}
\begin{eqnarray*}
C = \begin{bmatrix}
-1 & 0\\
-1 & 2 \\
1 & 0 \\
1 & -1 \\
0 & 1
\end{bmatrix}
\end{eqnarray*}
The condition to have a conservative system becomes 
\begin{eqnarray*}
\begin{bmatrix}
-1 & -1 & 1 & 1 &0\\
0 & 2 & 0 & -1 & 1
\end{bmatrix}
\begin{bmatrix}
w_1 \\
w_2\\
w_3\\
w_4\\
w_5
\end{bmatrix} = \begin{bmatrix}
-w_1-w_2+w_3+w_4 \\
2w_2-w_4+w_5
\end{bmatrix} = \begin{bmatrix}
0 \\
0
\end{bmatrix}
\end{eqnarray*}
which is verified if $w = (3,1,1,3,1)^T$. Here, the intuition is that we can produce two $X_1$ from one $X_1$ and another reactive $X_0$.

