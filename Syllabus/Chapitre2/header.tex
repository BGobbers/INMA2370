\documentclass[twoside,openright,11pt,a4paper]{report}

%\compiler avec xelatex
%\usepackage[applemac]{inputenc}
\usepackage[T1]{fontenc}
\usepackage[utf8]{inputenc} %latin1 est possible
%\usepackage[latin1]{inputenc} %latin1 est possible
\usepackage[francais]{babel}
\usepackage{lettrine}

\usepackage[text={13cm,20cm},centering]{geometry}

\renewcommand{\familydefault}{cmss}

\usepackage{graphicx}
\usepackage{amsmath}
\usepackage{amsfonts}
\usepackage{amssymb}
\usepackage{amsthm}
\usepackage{bm}
\usepackage{color}

\newcommand{\real}{\mathbb{R}}
\newcommand{\mb}{\mathbf}
\newcommand{\bos}{\boldsymbol}

\def \RR {I \! \! R}

\newcommand{\e}{\begin{equation}}  
\newcommand{\ee}{\end{equation}}
\newcommand{\eqn}{\begin{eqnarray}} 
\newcommand{\eeqn}{\end{eqnarray}} 
\newcommand{\eqnn}{\begin{eqnarray*}} 
\newcommand{\eeqnn}{\end{eqnarray*}} 

\newcommand{\bpm}{\begin{pmatrix}}
\newcommand{\epm}{\end{pmatrix}}

%\newcommand{\{\c c}}{\c c}

\newcommand{\bma}{\left(\begin{array}}
\newcommand{\ema}{\end{array}\right)} 
\newcommand{\hh}{\hspace{2mm}}
\newcommand{\hd}{\hspace{5mm}}
\newcommand{\hu}{\hspace{1cm}}
\newcommand{\vv}{\vspace{2mm}}
\newcommand{\vd}{\vspace{5mm}}
\newcommand{\vm}{\vspace{-2mm}}
\newcommand{\teq}{\triangleq}
%\newcommand{\qedb}{\,$\Box$}
\newcommand{\blanc}{$\left. \right.$}
\newcommand{\frts}[2]%
         {\frac{{\textstyle #1}}{{\textstyle #2}}}

\newcommand{\bindex}[3]%
{
\renewcommand{\arraystretch}{0.5}
\begin{array}[t]{c}
#1\\
{\scriptstyle #2}\\
{\scriptstyle #3}
\end{array}
\renewcommand{\arraystretch}{1}
}

\theoremstyle{definition}
\newtheorem{exemple}{{\bf Exemple}}[chapter]
\newtheorem{theoreme}[exemple]{{\bf Th{é}or{è}me}}
\newtheorem{propriete}[exemple]{{\bf Propri{é}t{é}}}
\newtheorem{definition}[exemple]{{\bf D{é}finition}}
\newtheorem{remarque}[exemple]{{\bf Remarque}}
\newtheorem{remarques}[exemple]{{\bf Remarques}}
\newtheorem{lemme}[exemple]{{\bf Lemme}}
\newtheorem{hypothese}[exemple]{{\bf Hypoth{è}se}}
\newtheorem{exercice}{{\bf Exercice}}[chapter]

\newcommand{\xqedhere}[2]{%
 \rlap{\hbox to#1{\hfil\llap{\ensuremath{#2}}}}}

\newcommand{\xqed}[1]{%
 \leavevmode\unskip\penalty9999 \hbox{}\nobreak\hfill
 \quad\hbox{\ensuremath{#1}}}

\newcommand{\gf}{\fg\,\,}

\newcommand{\cata}[1] %
     {\renewcommand{\arraystretch}{0.5}
     \begin{array}[t]{c} \longrightarrow \\ {#1} \end{array}
     \renewcommand{\arraystretch}{1}}

\usepackage[isu]{caption}
%\usepackage[font=small,format=plain,labelfont=bf,up,textfont=it,up]{caption}
\setlength{\captionmargin}{60pt}

\newcommand{\cqfd}
{%
\mbox{}%
\nolinebreak%
\hfill%
\rule{2mm}{2mm}%
\medbreak%
\par%
}

\pagestyle{headings}

\renewcommand{\sectionmark}[1]{%
\markright{\thesection.\ #1}{}}

\renewcommand{\chaptermark}[1]{%
\markboth{\chaptername\ \thechapter.\ #1}{}}

\makeatletter 
\def\@seccntformat#1{\csname the#1\endcsname.\;} 
\makeatother

\title{ {\Huge {\textbf{Modélisation et analyse  \\ \vspace{4mm} des systèmes dynamiques }}} \\ \vspace{4cm} G. Bastin}

%\title{ {\Huge {\textbf{Modelisation et analyse  \\ \vspace{4mm} des systemes dynamiques }}} \\ \vspace{4cm} G. Bastin}


\date{\today}